\section{单位制及量纲}\label{sec:04.04}

现在我们将已讲过的物理量及其单位列在表\ref{tab:04.03}\,中。在这些物
理量中,长度、时间及质量三者都是由规定的标准来作为单位的。
而其他所有的量的单位可以根据相应的定义公式,从这三个单位
来规定。如,速度的单位是由它的定义式\eqref{eqn:01.07.02}来规定的,$ G $
% 134.jpg
的单位是来自它的定义式
\begin{equation*}
	G = \frac { F r ^ { 2 } } { m _ { 1 } m _ { 2 } }
\end{equation*}
我们称长度、质量及时间三个物理量为基本量,它们的单位为基
本单位,称其他物理量为导出量,相应的单位为导出单位。
\begin{tablex}
	\caption{}
	\label{tab:04.03}
	\begin{tabularx}{\linewidth}{l|l|l|l}
		\toprule
		物理量 & \makecell[c]{单位(SI,MKS)} & \makecell[c]{单位(CGS)} & \makecell[c]{量\qquad 纲} \\
		\midrule
		长\quad 度 & 米 & 厘米 & $ L $ \\
		质\quad 量 & 千克 & 克 & $ M $ \\
		时\quad 间 & 秒 & 秒 & $ T $ \\
		速\quad 度 & 米/秒 & 厘米/秒 & $ L T ^ { - 1 } $ \\
		加速度 & 米/秒\.$^2$ & 厘米/秒\.$^2$ & $ L T ^ { - 2 } $ \\
		角速度 & 弧度/秒 & 弧度/秒 & $ T ^ { - 1 } $ \\
		力 & 牛顿 & 达因 & $ M L T ^ { - 2 } $ \\
		$G$ & 牛顿·米\.$^2$/千克\.$^2$ & 达因·厘米\.$^2$/克\.$^2$ & $ L ^ { 3 } M ^ { - 1 } T ^ { - 2 } $ \\
		\bottomrule
	\end{tabularx}
\end{tablex}

基本单位规定之后,整个一系列单位就被规定了。所以一组
基本单位就决定了一个单位制。我们经常采用的单位制,是长度
以米为单位,质量以千克(或叫做公斤)为单位,时间以秒为单位。
这个单位制称为国际单位制(SI)或米千克秒制(MKS)

另一种常用的单位制,是采用厘米为长度单位,克为质量单
位,秒为时间单位,称为CGS制。1厘米=1/100米,1克=1/1000千
克。有了这一组基本单位之间的换算关系,导出单位的换算关系
也就确定了。例如,在CGS制中速度单位与SI制中速度单位之
间的关系为\vspace{-0.56em}
\begin{equation*}
	1 \text{厘米/秒} = \frac { 1 } { 1 0 0 } \text{米/秒}
\end{equation*}
其他换算关系都可以类似求得。在数值运算中采用的单位制要统
% 135.jpg
一,不要一个量用这个单位制,另一个量又用那个单位制。

我们把上述讨论更一般化,如果取$ L $为长度“单位”,$ M $为
质量“单位”,$ T $为时间“单位”,则导出量的“单位”也可以用
$ L $,$ M $,$ T $表示,速度的“单位”就是$ L / T = L T ^ { - 1 } $ ,加速度是
$ L / T ^ { 2 } = L T ^ { - 2 } $,每个物理量都对应着一个由$ L $,$ M $及$ T $所组成的量,
这个量我们称为该物理量的量纲,从形式上可以说,物理量的量
纲,就是LMT制中的“单位”。我们习惯用一个方括号表示括号
中的物理量的量纲,如$ [ v ] = L T ^ { - 1 } $, $ [ a ] = L T ^ { - 2 }$。各量的量纲,也
列在表\ref{tab:04.03}\,中。

量纲很有用,无论是标量的加减运算,或是矢量的和差,所
涉及的量都应具有相同的单位,即应有相同的量纲。由此,一个
公式两边的量应具有相同的量纲。这个性质,常可以作为判别某
些不熟悉的量的量纲的方法。例如,利用式\eqref{eqn:04.03.04}可以求得$ G $的
量纲。由于$ [ m _ { 1 } ] = [ m _ { 2 } ] = M $,$ [ r ] = L $, $ [  F ] = M L T ^ { - 2 } $,要求式两
边的量纲相同,可得:
\begin{align*}
[ G ] &= \frac { [ F ] [ R ^ 2] } { [ m _ { 1 } ] [ m _ { 2 } ] } \\
&= \frac { M L T ^ { - 2 } L ^ { 2 } } { M ^ { 2 } } \\
&= M ^ { - 1 } L ^ { 3 } T ^ { - 2 }
\end{align*}

~ \vspace{-2em}

弧度是根据圆的弧长与其半径之比来定义的,所以它的量纲
是$ L / L = L ^ { 0 } $,即与$ M $,$ L $,$ T $均无关。这种量,称为无量纲量摩
擦系数也是一个无量纲量。

最后还应指出,可以不取长度、质量及时间三者为基本量,
而取另外三个量,如长度、时间及力,其他的量也都可以从长度、
时间及力三者导出,这样也可以规定单位制。这时,相当于由长
度$ L $、\!时间\,$ T $\,及力\,$ F $\,作为基本的量纲,其他量的量纲都可以用$ L $,
$ T $及$ F $表示出。选择不同的基本量,同一物理量的量纲就不同。
