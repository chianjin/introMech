\section[冲量]{冲\qquad 量}\label{sec:08.02}

体系能量的变化用外力作的功来描写,体系的动量的变化用
什么来描写?这就是本节将要讨论的问题。

根据牛顿第二定律,对于单个质点有
\begin{equation*}
    m \frac { \dif \vec{v} } { \dif t } = \vec{F}
\end{equation*}
因单个质点的动量为$  \vec{P} = m \vec{v}   $,故上述方程可写为
\begin{equation}\label{eqn:08.02.01}
    \frac { \dif \vec{P} } { \dif t } = \vec{F}
\end{equation}
即质点动量的变化率等于外力。

若质点在$  t _ { 1 }   $时,具有动量$\vec{P} _ 1$;在$ t _ { 2 } $时,变到$\vec{P} _ 2$,则由式 \eqref{eqn:08.02.01}
得
\begin{equation}\label{eqn:08.02.02}
    \vec{P} _ { 2 } - \vec{P} _ { 1 } = \int _ { t _ { 1 } } ^ { t _ { 2 } } \vec{F} \dif t
\end{equation}
即质点动量的变化,等于力对时间的积分。

$ \displaystyle \int _ { t _ { 1 } } ^ { t _ { 2 } } \vec{F} \dif t $
称为冲量。如果力为常量,则在$ t_1 $到$ t_2 $间的冲量可
写为$  \vec{F} \left( t _ { 2 } - t _ { 1 } \right) $。所以,冲量就是度量动量变化的物理量。

% 233.jpg
比较式\eqref{eqn:08.02.02}与式\eqref{eqn:06.02.08}\lhbrak 考虑到式\eqref{eqn:06.02.01}\rhbrak ,两者在形式
上是非常相似的,前者是力对时间的积分,后者是力对空间的积
分;前者反映出力与时间的联系,后者反映出力与空间的联系。
由此也可看到动量守恒与能量守恒二者的相同与不同。

作为动量守恒定律的一个应用,我们来讨论航天飞船问题。
飞船之所以能上天,是依靠火箭喷出气体的推动力。现在我们来
计算为要达到某一速度需要多少燃料的问题。乍看起来,这似乎
是个与燃料化学性质等有关的复杂问题,实际上,根据动量守恒,
可以给出一个基本关系。因为在应用动量守恒定律时,根本不需
顾及力的具体性质。
