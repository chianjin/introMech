\begin{questions}

\question 假设整个太阳系空间均匀地充满一种密度为$ \rho $的物质,它
对行星运动的阻力可忽略不计。试导出在此情形下行星按圆轨道
公转的周期与轨道半径之间的关系。推导时可认为太阳是质量为
$ M _ \text{ 日 } $的质点。

\question 如果太阳对周围行星的吸引力不是与$ r ^ {- 2} $成比例,而是与$ r ^ {- n} $
成比例。试问:对于作圆轨道运动的行星来说,开普勒第三定
律要作怎样的修改?

\question 在计算任意形状的物体$ A $与其外某一质点$ P $之间的万有引
力时,是否可以把A当作质量集中于其质心的质点来处理?

\question 任意形状的外壳对内部某一质点的引力都为零吗?

\question 中国传统的历法一—阴历,是以月亮绕地球的周期为月,
而且还有24个节气表示太阳在黄道上的位置,两个相邻节气之间
相隔$360 ^ { \circ } / 24 = 15 ^ { \circ }$  。由于12个月不是太阳在黄道上运行一周的时
间,因此在19年中有7个闰月,按规定,闰月中只能有一个节气。
试问:为什么闰月总是在夏季(即阴历5$ ~ $8月)?

\question 为什么中国发射的通信卫星要求定点于东经$70 ^ { \circ }$和东经
$125 ^ { \circ }$的赤道上空?

\question 在地球的引力场内,任何物体都受到地球引力的作用,作
% 150.jpg
向心的运动。但氢气球、水中的气泡等却有向上运动的趋势,它
们是否不受地球的引力?

\question 考虑到地球是球形,重力加速度的方向是指向地心的,抛
体的轨迹不应当是抛物线。那么,应当是什么曲线?

\question 考虑空气的摩擦力之后,抛体的运动方程中含有哪些与运
动物体本身物性有关的物理量?

\question 引力的“超距作用”观点意味着引力作用的传播速度是无
穷大,因而作用是瞬时发生的。近代物理理论认为,引力作用以有
限速率传播,所以经典理论应当作相应的修改。如果你有兴趣,
可以参考 Jonothan L. Logan, \textit{Gravitational Wave - Progress Report} , Physics Today, 1973.3;以及 I. J. Good, \textit{Infinite Speed of Propagation of Gravitation in Newtonian Physics}, American Journal of Physics, 1975.7。

\end{questions}