\section{曲线运动的速度}\label{sec:01.07}

我们现在来推广上节关于直线运动的速度的概念。按图1.11,
质点作曲线运动,在时刻$t_1$,质点位置矢量为$\vq{r}(t_1)$,在时刻$t_2$,
位置矢量为$\vq{r}(t_2)$,则定义在$t_1$到$t_1$间隔中质点A的平均速度为
\begin{equation}\label{eqn:01.07.01}
    \langle v \rangle_{t_1\rightarrow t_2}=\frac{\vq{r}(t_2)-\vq{r}(t_1)}{t_2-t_1}=\frac{\Delta \vq{r}}{\Delta t}
\end{equation}
将其与式(1·6·1)对比可见。只是将其中的$x(t)$换成了相应的
$\vq{r}(t)$。在(1·7·1)中,$\Delta t=t_2-t_1$,$\vq{r}=\vq{r}(t_2)-\vq{r}(t_1)$,
后者是$t_1$到$t_2$间隔中质点位置矢量的改变量,称为位移矢量。
这个平均速度的定义表明,平均速度是矢量。根据矢量运算
规则,式(1·7·1)所定义的$\langle v\rangle t_1-t_2$是在l到2的方向(图1-11)。
另一方面,由图1·ll很清楚知道,在$t_1$到$t_2$时间内质点$A$的运
动方向并非总是沿着l到2的方向的,而是先从1向3方向运动,
然后从3向2方向运动,这些运动方向并不平行于1到2的方向。
所以平均速度所指的方向,只是质点$A$真实运动方向的平均。
也就是说,平均速度不但对于运动快慢的描写是粗
略的,而且对于运动方向的描写也是粗略的。

对式\eqref{eqn:01.07.01}取$\Delta t \rightarrow 0$的极限,就得到瞬
时速度的定义。
\begin{equation}\label{eqn:01.07.02}
    \begin{aligned}
        \vq{v}(t) & =\lim _{\Delta t \rightarrow 0} \frac{\vq{r}(t+\Delta t)-\vq{r}(t)}{\Delta t} \\
                  & =\lim _{\Delta t \rightarrow 0} \frac{\vq{r}}{\Delta t} =\frac{d\vq{r}}{dt}
    \end{aligned}
\end{equation}
它是直线运动情况的式\eqref{eqn:01.06.03}的推广。如果利用式\eqref{eqn:01.04.01},则
有
\begin{equation}\label{eqn:01.07.03}
    \vq{v}(t)=\frac{d x(t)}{d t} \vq{i}+\frac{d y(t)}{d t} \vq{j}+\frac{d z(t)}{d t} \vq{k}
\end{equation}
所以三个坐标函数$x(t)$,$y(t)$,$z(t)$对时间t的导数分别是速度矢
量在三个坐标轴方向的分量:
\begin{equation}\label{eqn:01.07.04}
    v_{x}=\frac{d x(t)}{d t}, ~ v_{y}=\frac{d y(t)}{d t}, ~ v_{z}=\frac{d z(t)}{d t}
\end{equation}

由图l·12,当$\Delta t$减小时,矢量$\vq{r}(t+\Delta t)$
逐渐向$\vq{r}(t)$靠拢,矢量$\Delta \vq{r}$相继从1,2变到1,3,
变到1,4……,在$\Delta t ightarrow 0$的极限情况下,$\Delta \vq{r}$
的方向趋于轨迹曲线在点1的切线方向。这样,我们就得到一个
结论:瞬时速度$v(t)$的方图l·12曲线运动的瞬时速度
向,就是轨迹曲线在相应点的切线方向。瞬时速度的大小$v$,按
定义\eqref{eqn:01.07.02}应为
\begin{equation*}
    \begin{aligned}
        v\equiv |\vq{v}| & =\left|\lim _{\Delta t \rightarrow 0} \frac{\Delta \vq{r}(t)}{\Delta t}\right| \\
                         & =\lim _{\Delta t \rightarrow 0} \frac{|\Delta \vq{r}|}{\Delta t}
    \end{aligned}
\end{equation*}
我们可以引入路程长度概念来描写运动。如图1·13,若当$t=0$时,
质点在轨迹上的$P$处。则可定义函数$s(t)$,它表示质点到$t$时刻所
走过的路程的长度。显然,$\Delta s=s(t+\Delta t)-s(t)$表示
在$t$到$t+\Delta t$中质点所走过的路程的长度。一般$\Delta s$和
$|\Delta \vq{r}|$并不相等,但在$\Delta t \rightarrow 0$的极限情况
下,二者是一样的,故速度大小也可表示为
\begin{equation}\label{eqn:01.07.05}
    v=\lim_{\Delta t \rightarrow 0}\frac{\Delta s}{\Delta t}=\frac{ds}{dt}
\end{equation}

