\section{几种速度分布解}\label{sec:05.03}

我们再来讨论速度分布,仍利用图51。设O观察者所得到的
速度为 q \left( r , t \right)  ,其意义是在t时刻相对于0位于处的天体的平
均速度。类似地,O所观测到的速度取为 q ^ { \prime } \left( F ^ { \prime } , t \right)
由哥白尼原理,O'观测到的位于r的天体速度应等于O观测
到的位于r=r的速度,即
g ^ { \prime } \left( r ^ { \prime } , t \right) = q \left( f ^ { \prime } , t \right)
(5.31)
如果再让O及O观测同一点P,则按照牛顿运动学中的速度合成
律,应有
q ^ { \prime } \left( r ^ { \prime } , t \right) = q \left( r , t \right) - \frac { d c } { d t }
(5.32)
\left( 5 \cdot 2 \cdot 3 \right) , \frac { d c } { d t }
其中r及r同样满足式是O相对于0的速度。故有
\frac { d c } { d t } = q \left( c , t \right)
\left( 5 \cdot 3 \cdot 3 \right)
从式 \left( 5 \cdot 3 \cdot 1 \right)  、 \left( 5 \cdot 3 \cdot 2 \right)  、 \left( 5 \cdot 3 \cdot 3 \right)  、 \left( 5 \cdot 2 \cdot 3 \right)  可得
g \left( t - c , t \right) = g \left( r , t \right) - q \left( c , t \right)
\left( 5 \cdot 3 \cdot 4 \right)
这表明 q \left( r ,  t)是的线性函数,它只有以下形式的解:
g \left( r , t \right) = f \left( t \right) t
\left( 5 \cdot 3 \cdot 5 \right)
这样,宇宙的物质速度分布只可能有三种形式:
1.膨胀解
f \left( t \right) > 0  ,天体做均匀膨胀的运动,膨胀速度正比于相对于
观察者的距离。
2.收缩解
f \left( t \right) < 0  ,天体做均匀收缩的运动,收缩速度正比于相对于观
察者的距离。