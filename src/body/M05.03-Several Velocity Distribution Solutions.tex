\section{几种速度分布解}\label{sec:05.03}

我们再来讨论速度分布,仍利用图\ref{sec:05.01}。设$ O $观察者所得到的
速度为$ \vec{ q } \left( \vec{r} , t \right) $,其意义是在$ t $时刻相对于$ O $位于$ \vec{r} $处的天体的平
均速度。类似地,$ O' $所观测到的速度取为$ \erratanote{\ensuremath{\vec{ q }'}}{$ q '$} \left( \vec{r} ', t \right) $。

由哥白尼原理,$ O' $观测到的位于$\vec{r}'$的天体速度应等于$ O $观测
到的位于$ \vec{r} = \vec{r}' $的速度,即
\begin{equation}\label{eqn:05.03.01}
 \vec{q} ' \left( \vec{r} ' , t \right) = \vec{q} \left( \vec{r} ' , t \right)
\end{equation}
如果再让$ O $及$ O' $观测同一点$ P $,则按照牛顿运动学中的速度合成
律,应有
\begin{equation}\label{eqn:05.03.02}
 \vec{q} ' \left( \vec{r} ' , t \right) = \vec{q} \left( \vec{r} , t \right) - \frac{ \dif \vec{c} }{\dif t}
\end{equation}
其中$ \vec{r}' $及$ \vec{r} $同样满足式\eqref{eqn:05.02.03},$ \dfrac{ \dif \vec{c} }{\dif t} $是$ O' $相对于$ O $的速度。故有
\begin{equation}\label{eqn:05.03.03}
 \frac { \dif \vec{c} } { \dif t } = \vec{q} \left( \vec{c} , t \right)
\end{equation}
从式\eqref{eqn:05.03.01} 、\eqref{eqn:05.03.02} 、\eqref{eqn:05.03.03}、\eqref{eqn:05.02.03}可得
\begin{equation}\label{eqn:05.03.04}
 \vec{q} \left( \vec{r} -\vec{c} , t \right) = \vec{q} \left( \vec{r} , t \right) - \vec{q} \left( \vec{c} , t \right)
\end{equation}
这表明$ \vec{q} \left( \vec{r} , t \right) $是$ \vec{r} $的线性函数,它只有以下形式的解:
\begin{equation}\label{eqn:05.03.05}
 \vec{q} \left( \vec{r} -\vec{c} , t \right) = f \left(t\right) \vec{r}
\end{equation}
这样,宇宙的物质速度分布只可能有三种形式:

{\heiti 1.膨胀解}

$ f \left( t \right) > 0 $,天体做均匀膨胀的运动,膨胀速度正比于相对于
观察者的距离。

{\heiti 2.收缩解}

$ f \left( t \right) < 0 $,天体做均匀收缩的运动,收缩速度正比于相对于观
察者的距离。

% 159.jpg
\clearpage
{\heiti 3.静止解}

$ f \left( t \right) = 0 $,天体之间都相对静止。
