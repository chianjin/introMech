\section{简谐振动的合成}\label{sec:07.06}

简谐振动只是振动或周期运动中的一种,许多实际的周期运
% 214.jpg
动并不是简谐的。例如,各种乐器的振动大都不是简谐振动。所
谓音色,就是决定于振动的形式。譬如,小提琴的振动有如图
710所示的锯齿形。各种乐器都有自己的独特振动形式。

图7.10小提琴的振动

这种情况,似乎要求我们把振动按其形式分为多种类型来建
立理论。其实不然,因为各种周期振动都可以由简谐振动组成。
理论上可以证明:任意一个周期振动$ x = f ( t ) $,即$ f ( t ) $是$ t $的周期
函数,以$ T $为周期,即
\begin{equation}\label{eqn:07.06.01}
	f ( t + n T ) = \erratanote{\ensuremath{f ( t )}{原文“$f(T)$”。(\ref{sec:07.06}节)}
\end{equation}
其中$ n $为任意正、负整数,则$ f ( t ) $可以表示成
\begin{equation}\label{eqn:07.06.02}
	\begin{aligned}
		f ( t ) =& A _ { 0 } + A _ { 1 } \cos ( \omega t + \vraphi _ { 1 } ) + A _ { 2 } \cos ( 2 \omega t + \varphi _ { 2 } ) \\
		&+ A _ { 3 } \cos ( 3 \omega t + \varphi _ { 3 } ) + \cdots  
	\end{aligned}
\end{equation} 
其中$\omega = \frac { 2 \uppi } { T } $;$ A _ { 0 }, A _ { 1 }, A _ { 2 }, \dots $决定于$ f ( t ) $的具体形式。式\eqref{eqn:07.06.01}
及式\eqref{eqn:07.06.02}表示,任何一个周期为$ T $的振动,可以看成是由周期
为$ T, \dfrac { 1 } { 2 } T , \dfrac { 1 } { 3 } T  , \dots $一系列简谐振动相加而成的。例如上述小
提琴的振动可写为
\begin{equation*}
	x = A ( \sin \omega t + \frac { 1 } { 2 } \sin 2 \omega t + \frac { 1 } { 3 } \sin 3 \omega t + \cdots ) 
\end{equation*}
所谓音色,就决定于$ A _ { 1 } , A _ { 2 }, \dots $各项简谐振动的振幅之比例。声
% 215.jpg
音是否和谐,也就决定于这些比例。小提琴的一系列振幅的比是
$ \dfrac { 1 } { 1 } : \dfrac { 1 } { 2 } : \dfrac { 1 } { 3 }  
: \cdots $非常简单而有规律,这就是小提琴声音优美动听的
物理原因。

现在讨论几种简单的振动合成问题。

(1)一种最简单的情况是
\begin{equation}\label{eqn:07.06.03}
	x = A _ { 1 } \cos ( \omega t + \varphi _ { 1 } ) + A _ { 2 } \cos ( \omega t + \varphi _ { 2 } )  
\end{equation}
这表示质点的运动是两种简谐振动合成的结果,而且这两种简谐
振动具有相同的周期。

利用三角公式,可以将上式改写成
\begin{equation}\label{eqn:07.06.04}
	\begin{aligned}
		x =& ( A _ { 1 } \cos \varphi _ { 1 } + A _ { 2 } \cos \varphi _ { 2 } ) \cos \omega t \\ 
		&- ( A _ { 1 } \sin \varphi _ { 1 } + A _ { 2 } \sin \varphi _ { 2 } ) \sin \omega t  
	\end{aligned}
\end{equation}
( 7 \cdot 6 \cdot 4 )  
令
\begin{equation}\label{eqn:07.06.05}
	\begin{aligned}
		A _ { 1 } \cos \varphi _ { 1 } + A _ { 2 } \cos \varphi _ { 2 } = A  \cos \varphi  \\
		A _ { 1 } \sin \varphi _ { 1 } + A _ { 2 }  \sin \varphi _ { 2 } = A  \sin \varphi
	\end{aligned}  
\end{equation}
则式\eqref{eqn:07.06.04}成为
\begin{equation}\label{eqn:07.06.06}
	x = A  \cos ( a t +  \varphi )  
\end{equation}
( 7 \cdot 6 \cdot 6 )  
由式 ( 7 \cdot 6 \cdot 5 )  ,A及φ可以表示成
A = \sqrt { A _ { 1 } ^ { 2 } + A _ 2 ^ { 2 } + 2 A _ { 1 } C _ { 2 }  \cos (  \varphi _ { 1 } -  \varphi _ { 2 } ) }  
( 7 \cdot 6 \cdot 7 )  
g R = \frac { A _ { 1 }  \sin \varphi _ { 1 } + A _ 2 ^ { 2 } } { A _ { 2 } }  \sin \varphi _ { 2 } 
式 ( 7 \cdot 6 \cdot 6 )  表示,两个相同周期(或频率)的简谐振动合成的结果
仍是一个简谐振动,并具有
相同周期,只是振幅和位相
应取式 ( 7 \cdot 6 \cdot 7 )  给出的值。
式 ( 7 \cdot 6 \cdot 7 )  )也很容易由
几何方法得到。按73节的
方法,每个简谐振动可用一
矢量表示,则式 ( 7 \cdot 6 \cdot 3 )  应由
图7.11振动的合成
% 216.jpg
图7.11中的矢量0A1及0A2表示。由于 O A _ { 1 }  ,OA2绕O转动的角速
度(即)相同,所以二者相对位相不变,因此二者在OX方向的
投影之和,就等于矢量OA的投影。OA为OA1及OA2的矢量和。
由 A _ { n }  , A _ { 2 }  ,甲1,2可求出A及φ,结果仍然是式(7.6.7)。
(2)两个振幅相同,但周期差别很小的简谐振动的合成,即
x = A  \cos ( \omega _ { 1 } t + \varphi ) + A  \cos ( \omega _ { 2 } t + \varphi )  
( 7 \cdot 6 \cdot 8 )  
其中a1与 \omega _ { 2 }  的差很小。
| \omega _ { 1 } - \omega _ { 2 } | < \omega _ { 1 } , \omega _ { 2 }  
利用三角公式,式(76.8)可以改写成
x = 2 A  \cos ( \frac { \omega _ { 1 } - \omega _ { 2 } } { 2 } t )  \cos ( \frac { \omega _ { 1 } + \omega } { 2 } \omega _ { 2 } t + \varphi )  
rightarrow 2 A  \cos ( \frac { \omega _ { i } } { \omega _ { 2 } } t )  \cos ( \omega _ { 1 } i + \varphi )  
这个公式的物理意义是,质点仍作频率为1的简谐振动,但它的
振幅不是常数,而是 | 2 A  \cos \frac { \omega _ { 1 } - \omega _ { 2 } } { 2 } t |  ,即振幅也是随时间周期
变化的。振幅的平方正比于振动能量,所以,这种振动具有周期
生的强弱变化,这种现象称为拍。振幅的变化周期即为拍的周期。
由于振幅只涉及  \cos \frac { \omega _ { 1 } - \omega _ { 2 } } { 2 } t  的绝对值,所以,单位时间中的振
\frac { 1 } { 2 \pi } - \frac { | \omega _ { 1 } - \omega _ { 2 } | } { 2 }  
福变化次数等于的两倍,故拍的周期为
T \ne = \frac { 2 \pi } { | \omega _ { 1 } - \omega _ { 2 } } 
(3)二维的振动合成。由73节的讨论已知,一个匀速圆周运
力可以看成在X,Y两个方向上的简谐振动的合成,即匀速圆周
运动可表示为
% 217.jpg
y = \frac { v _ { 2 }  \sin ( \omega t + \varphi _ { 0 } ) } { v _ { 2 }  \sin ( \omega t + \varphi _ { 0 } ) }  
( 7 \cdot 6 \cdot 9 )  
其中为圆周的半径。
现在我们把式 ( 7 \cdot 6 \cdot 9 )  加以推广。假定x方向与y方向振动的
位相不一定相同,即
x = \frac { r _ { 2 }  \cos ( \omega t + \varphi _ { 1 } ) } { v _ { 2 }  \cos ( \omega t - \varphi _ { 2 } ) }  
( 7 \cdot 6 \cdot 1 0 )  
这时,运动的形态完全取决于 \varphi _ { 1 }  , \varphi _ { 2 }  的相互关系。
如果 \varphi _ { 2 } - \varphi _ { 1 } = 3 \pi / /  ,则式 ( 7 \cdot 6 \cdot 1 0 )  变为式 ( 7 \cdot 6 \cdot 9 )  ,它是匀速
圆周运动。
如果  \varphi _ { 2 } - \varphi _ { 1 } = 0  ,则式 ( 7 \cdot 6 \cdot 1 0 )  )成为
\pi = \pi _ { 0 }  \cos ( \omega t + \varphi _ { 1 } )  
y = r _ { 0 }  \cos ( \omega t +  \varphi _ { 1 } )  
因此,这时的轨迹是一段直线:
x = y  , | x |  , | y | < r _ { f } 
如果定义坐标
r = \sqrt { x ^ { 2 } + y ^ { 5 } }  
则
r = \sqrt { 2 } r _ { 0 }  \cos ( \omega t +  \varphi _ { i } )  
故质点是沿着图712(a)的斜线方向作简谐振动。
为了讨论一般情况,我们可以 \hat { r } ( 7 \cdot 6 \cdot 1 0 )  得出
x + y = \gamma _ { 0 }  \cos ( \omega t + \varphi _ { 1 } ) + v _ { 0 }  \cos ( \omega t +  \varphi _ { 2 } )  
= A _ { + }  \cos ( \omega t + \varphi ) 
( 7 \cdot 6 \cdot 1 1 )  
x - y = r _ { 0 }  \cos ( \omega t +  \varphi _ { 1 } ) - r _ { 0 }  \cos ( \omega t +  \varphi _ { 3 } )  
A B = A _ { 4 }  \sin ( \omega t + \varphi )  
其中
A _ { + } = \sqrt { 2 } r _ { 0 } \sqrt { 1 +  \cos ( \sqrt { l } -  \varphi _ { 2 } ) }  
A _ { - } = \sqrt { 2 } r _ { 0 } \sqrt { 1 - c }  \sin (  \varphi _ { 1 } -  \varphi _ { 3 } )  
(76.12)
t g \varphi = \frac {  \sin \varphi \varphi _ { 1 } } {  \cos \varphi _ { 1 } } +  \cos \varphi _ { 1 } 
% 218.jpg
由式 ( 7 \cdot 6 \cdot 1 n )  即得
\frac { ( x + y ) ^ { 2 } } { b } + \frac { ( x - y ) ^ { 2 } } { 4 } = 1  
可见,式 ( 7 \cdot 6 \cdot 1 0 )  表示的运动轨迹,一般情况是一个椭圆。椭圆
的长轴或短轴沿着±45°线,两轴长分别为A+及A。由此,不难
从位相差2-φ1给出一系列不同的运动形态(图7.12)。
\frac { 5 } { 4 } - \pi  
\frac { 3 } { 2 } x   \frac { 7 } { 4 } \pi   \frac { 3 } { 2 } x - 1 0 ^ { \frac { 3 } { 4 } }   \frac { 7 } { 4 } \pi 
(
()
图7.12二维振动的合成