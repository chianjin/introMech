\section{相对和绝对}\label{sec:02.01}

人对自然界认识的深化,常常是和弄清什么是相对的、什么
是绝对的这类问题联系在一起的。

远古时期,无论在东方文明或西方文明中,都认为大地是平
坦的,天在大地的上面。“天圆地方”就是这种观念的通俗表
述。用现代语言来说,在这种观念中,“上”“下”这两个方向是
绝对的。

到古希腊时期,毕达哥拉斯以及亚里士多德等先后开始主张
大地是一个球体,即地球。中国的“浑天说”也有大体相似的观
念。这是认识上的一次进步,因为它抛弃了当时的一种“习惯”
的但却不正确的观念——“上”“下”是绝对的。

按照当时“习惯”的看法,如果大地是球形,那些居住在我们
的对蹠点上的人不是早就“掉下”去了吗?可见,树立球形大地
观需要克服一些不正确的成见所带来的阻力。因此,从相对与绝
对角度来评价,可以说,地球观是把“上”和“下”这两个方向
相对化了。我们看对照点的人在“下”,对照点的人看我们也是
在“下”,亦即空间各个方向是等价的,没有一个方向具有特别
的绝对优越的性质。

在亚里士多德的体系中,认为地球的球心是宇宙的中心。这
个位置具有非常特殊的,绝对的意义。亚里士多德还认为,物体
运动的规律是力图达到自己的天然位置,地面附近物体的天然位
置就是地球的中心,远处物体(如星体)则应环绕着地球的中心。
这样,在支配物体运动的规律中,空间位置具有特别的作用,这
种性质,可以叫做空间位置的绝对性。

以牛顿力学为起点的物理学,否定了亚里士多德体系中的空
间位置的绝对性,认为任何的空间点都是平权的,地心在宇宙中
并不占有特殊的地位。牛顿理论中的相对和绝对,又不同于亚里
士多德了。

正因为相对和绝对这一问题的重要性,在这一章里,我们将
系统地分析一下牛顿的运动学中的相对性,并且还将指出,牛顿
体系中的相对绝对观也是有局限的,在某些条件下,就完金不适
用了。