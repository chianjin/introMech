\section{简谐振动的几何表述}\label{sec:07.03}

为了进一步理解简谐振动\lhbrak 式\eqref{eqn:07.02.04}\rhbrak 的特性,我们再来研
究匀速圆周运动。如图76表示一个以$ r _ { 0 } $为半径的圆周运动。若角
速率为恒定的,则按\ref{sec:01.09}\,节中的表示可知,质点的角位置$ \varphi $与角速
率$ \omega $的关系为:
\begin{equation}\label{eqn:07.03.01}
	\varphi = \omega t + \varphi _ { 0 }
\end{equation}
式中$ \varphi _ { 0 } $是$  t = 0   $时刻的角位置。

图76圆周运动与简谐振动

如果我们用$ \left(x, y\right) $坐标来描写质点的位置则有
\begin{equation}\label{eqn:07.03.02}
	\begin{aligned}
	x &= r _ { 0 } \cos \varphi  \\
	y &= r _ { 0 } \sin \varphi
	\end{aligned}
\end{equation}
将式\eqref{eqn:07.03.01}代入式\eqref{eqn:07.03.02},得到
\begin{equation}\label{eqn:07.03.03}
    \begin{aligned}
    x &= r _ { 0 } \cos \left( \omega t + \varphi _ { 0 } \right) \\
    y &= r _ { 0 } \sin \left( \omega t + \varphi _ { 0 } \right)
\end{aligned}
\end{equation}
对比式\eqref{eqn:07.03.03}及式\eqref{eqn:07.02.04},可见两者是完全相同的,只是式
\eqref{eqn:07.02.04}中的振幅$ A $在式\eqref{eqn:07.03.03}中变成了$ r _ { 0 } $。因此,结论是:匀速
圆周运动质点的$ x $坐标按简谐振动方式变化。

这个结果启发我们可以把每个简谐振动用一个匀速
圆周运动来表述。如图7.7,如果简谐振动是沿着$ X $轴
的,那么我们可以用矢量$ \vec{ OA } $
来代表这个简谐振动。$ \vec{ OA } $
的长度等于简谐振动的振幅,它以匀角速率$ \omega $绕$ O $转动。$ \vec{ OA } $与
$ X $轴的夹角$ \varphi $,就是位相。矢量$ \vec{ OA } $在$ X $轴上的投影即为振子的
位置坐标,这种表述简谐振动的方法称为几何表述。

图77简谐振动的几何表述