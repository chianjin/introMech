\begin{exercises}

\exercise 5.0公斤的物体放在地面上,若物体与地面之间的摩擦系
数是0.30,至少要多大的力才能拉动该物?

\exercise 一个5.0吨的空车箱能使车箱底下的弹簧压缩1.0厘米。如
果在载货后弹簧压缩4.0厘米,求货的质量。

\exercise 两人拉纤使船前进,两纤绳互成直角,他们分别用力20公斤和15公
斤(图322),求使船前进的合力。

\exercise 质量为$ 1.3 \times 10^4 $公斤的火箭
上升时,向上的推力为$ 2.6 \times 10 ^ { 5 } $牛顿,

图322

试求它的加速度。

\exercise 一车的质量为$ 2.85 \times 10 ^ { 5 } $公斤,当车以30公里/小时的速度
行驶时,作紧急刹车。已知制动阻力是车重的0.6倍,求刹车距离
(即刹车后所走的距离)。

\exercise 在电梯中放一磅秤,一个50公斤的人站在磅秤上。求:

(1)当电梯匀速上升或下降时,磅秤指示是多少?

(2)当电梯以4.9米/秒2的加速度上升或下降时,磅秤的指示各为多少?

\exercise 如图3.23所示,用定滑轮将质量为$ m $的重物送往高处,人
的质量为$ M $,绳不可伸长,绳的质量及它与滑轮的摩擦可忽略。
% 114.jpg
当物体$ m $以匀速上升及以加速度$ a $上升时,人对地面的压力各是多
少?

图3·23
图324

\exercise 如图3.24所示,已知$ m _ { 1 } = 3 . 0 $公斤,$ m _ { 2 } = 2 . 0 $公斤,试求

(1)每一物体的加速度;

(2)~~$ m _ { 2 } $受到的绳子拉力。

\exercise 图3.25的装置可用来测物体$ A $与桌面间的摩擦系数$ \mu $。设
已知$ A $,$ B $的质量分别是$ m_A $和$ m_B $,它们的加速度是$ a $,试导出
摩擦系数的表达式。

\exercise 两人分别将一小车以同样的加速度推上坡,一人的推力
方向与斜面平行,以$ F_1 $表示;另
一人的推力方向与水平面平行,以$  F_2  $表示。设车与斜面的摩擦系
数$ \mu $及斜面的倾角$ \alpha $已知,求两人推力之比。

图325

\exercise 用起重机吊起一个4吨重的物件,吊索最多可承受5.0吨
的拉力,吊索本身重量可不计,求在下列各情况中吊索所承受的
拉力:

(1)物件吊在空中静止;

(2)物件以25厘米/秒的速度匀速上升;

(3)物件以80厘米/秒的速度匀速下降;
% 115.jpg

(4)物件由静止匀加速上升,在0.5秒内速度增加到20厘米/
秒;

(5)物件由40厘米/秒的速度开始匀加速下降,0.5秒内增加
到80厘米/秒;

(6)要使吊索不断,物体向上的最大加速度是多少?

\exercise 质量$  m _ { 1 } = 3 0 0  $克的物体
与$  m _ { 2 } = 2 0 0  $克的物体通过定滑轮
用绳连接起来(图326)。物与水
平桌面间摩擦系数$  \mu = 0 . 2 5  $,桌子
不动,绳长不变,绳的质量及绳与滑轮的摩擦可略去。这系统的加
速度是多少?绳的张力是多少?

图326

若互换$ m_1 $与$ m_2 $,有无影响?

\exercise 将质量分别为$ m_1 $, $ m _ { 2 } $ , $m _ { 3 }$ 和$ m $的四个物体连接(图3·27)
桌面与这些物体之间的摩擦系数都是$ \mu $。设绳长不变,桌子与滑
轮位置不变,绳子的质量及绳与滑轮间的摩擦可忽略不计。求这
系统的加速度以及各物体之间的张力$  T _ { 1 }  $,$  T _ { 2 }  $,$  T _ { 3 }  $。

图327

\exercise 质量分别为$ m_1 $,$  m _ { 2 } $ 和$ m_3 $的三个物体($ m _ { 2 } > m _ { 1 } $),用绳子按
图3.28所示连接。$ A $和$  B  $是两个轻滑轮。设斜面和滑轮位置不变,绳
长不变,略去摩擦力和绳子的质量。求系统的加速度$ a $和绳中的
% 116.jpg
张力$  T _ { 1 }  $,$  T _ { 2 }  $;$ A  $点承受的力;绳$ d $上的张力。

图328
\exercise 一个学生要确定一个盒子与一块乎板之间的静
摩擦系数$ \mu _ 0 $及滑动摩擦系数 $\mu$。他把盒子放在平板上,
渐渐抬高板的一端,当板的倾角(即板与水平之夹角)达
\ang{30}时,盒子开始滑动,并
恰好在4.0秒内滑下4.0米距离。试由这些数据确定 $\mu_0$及 $\mu$。

\exercise 如图3.29,质量为$ M $的三角形斜面上放一个小质量物体$ m $,
三角形物体放在水平面上,假设所有接触都是光滑的。求:

图329

(1)必须用多大的水平推力$ F $,才能使$ m $相对于$ M $为静止?

(2)此时系统的加速度$ a $有多大?

\exercise 收尾速度问题。空气对物体的阻力由许多因素决定。然
而,一个有用的近似公式是,阻力$  \vec{f} _ { p } = - \beta \vec{v}  $,其中$\vec{v}$是物体的速度,
$\beta$是一个与速度无关的常数。现在考虑空气中的一个自由下落物
体,将$ Z $轴的正方向取为竖直向下。

(1)给出落体的牛顿方程。

(2)当物体的速度$  v( t _ 0)  $等于多少时,物体不再加速(这个速
度叫做收尾速度)?

(3)试证,速度随时间变化的关系为:
\begin{equation*}
    v ( t ) = v ( t _ 0 ) \left( 1 - e ^ { - \frac { \beta } { m } t } \right)
\end{equation*}
并作出$ v\mbox{-}t $曲线。

\end{exercises}