\section{牛顿第三定律}\label{sec:03.03}

在前两节中,我们讨论了牛顿力学的前两个定律,即惯性定
律和动力学的基本方程。从动力学的角度来说,有了这些定律就
已完整了。牛顿第三定律实际上是关于力的性质的定律,而不是
动力学本身的定律。

牛顿给出了物体间各种作用力所共有的一种性质,即作用等
于反作用。详细些说,物体间作用力总是成对出现的,如果质点
$ A $对质点$ B $的作用力为$ \vec{F}_{A \rightarrow B}$,那么,质点$ B $对质点$ A $也有作用力
$ \vec{F}_{B \rightarrow A}$,而且两个力的大小相等,方向相反,并位于两质点的连
线上,即 \vspace{-0.19em}
\begin{equation}\label{eqn:03.03.01}
    \vec{F}_{A \rightarrow B} = -\vec{F}_{B \rightarrow A}
\end{equation}
这条定律,称为牛顿第三定律。
% 096.jpg

第三定律也有一定的应用范围。动力学的基本方程式\eqref{eqn:03.02.01}
是在存在惯性系的基础上建立起来的。同样,第三定律也是建立
在惯性系的基础上。另外要强调指出,即使在惯性系中,第三定
律也是有时对,有时并不对。若物体之间彼此接触才有相互作用
力,我们称为接触力。对于接触力,第三定律总是成立的。但是,
对于两物体间有一定距离时的相互作用的力,第三定律有时成立,
有时不成立。譬如,两个电荷之间的电磁作用力,第三定律就不
总是正确的。