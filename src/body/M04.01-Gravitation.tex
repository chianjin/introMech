\section{开普勒的行星运动三定律}\label{sec:04.01}
% 123.jpg
牛顿第二定律\lhbrak\,式\eqref{eqn:03.02.01}\,\rhbrak\,是物体运动的动力学基本规律。然
而,如果没有关于力的详细知识,这个规律是不能应用的。在这
个意义上,我们可以说牛顿第二定律的含义是:研究物体的动力
学的关键是研究物体之间的相互作用力。牛顿在《原理》一书中
曾写道:“我奉献这一作品,作为哲学的数学原理,因为哲学的
全部责任似乎在于——从运动的现象去研究自然界中的力,然后
从这些力去说明现象。”

牛顿使用“从运动的现象去研究力,从力去说明现象”的方
法建立了万有引力定律。

在牛顿之前,人类研究得最多也最清楚的运动现象就是行星
的运行。肉眼可以看到五颗行星:水、金、火、木、土。对这五
颗行星的运动有过长期的观察,特别是丹麦天文学家第谷连续进
行了二十年的仔细观测,他的学生开普勒则花费了大约二十年的
时间分析这些数据。开普勒前后总结出三条行星运动的规律:

(1)所有行星都沿着椭圆轨道运行,太阳则位于这些椭圆的
一个焦点上。这称为轨道定律。

(2)任何行星到太阳的连线在相同的时间内扫过相同的面
积。这称为面积定律
% 124.jpg

(3)任何行星绕太阳运动的周期的平方与该行星的椭圆轨道
的半长轴的立方成正比,即
\begin{equation}
	T \propto r ^ { 3 / 2 }  
\end{equation} 
式中,$ T $是行星运动的周期;$ r $是椭圆轨道的半长轴。这称为周期
定律。