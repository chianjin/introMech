\section[参考系]{参~~考~~系}
    牛顿力学所研究的对象是物体的机械运动。从我们日常见到
的车行马跑,及至大到月亮、太阳等星体的运行,小到分子、原子、
粒子的一些飞行,都是属于这一类运动。这类运动的共同特点,
就是物体在空间的位置时刻在变化着。当然,静止的状态、平衡
的状态也是力学的内容之一。

    牛顿意义下的运动学,就是研究如何描写物体位置的变化。

\renewcommand{\hsp}{\hspace{0.1em}}
    研究问题总是从简单的情况入手。我们首先讨论一种被称为
质点的物体,即大小为零的物体。我们知道,任何具体的物体总
是有一定大小的,没有任何一个物体与质点完全等价。但是,对
于某些特定的运动来说,可以足够准确地把物体看作一个质点。
譬如,在讨论地球绕太阳的公转时,由于地球的半径\hsp(约\hsp 6,400\hsp 公
里)\hsp 比地球与太阳的距离\hsp (约\hsp 149,504,000\hsp 公里)\hsp 小得多,把地球作
为质点是相当好的近似,或者说,在此情况下,将地球作为质点
处理,是一个足够准确的模型。显然,这种模型是有一定适用限
度的。当讨论到地表问题时,再把地球看作质点就大谬不然了。

其他文字其他文字其他文字其他文字其他文字其他文字
\begin{align}
    \vq{r}=x\vq{i}+y\vq{j}+z\vq{k}\\
    \vq{r}=x\vq{i}+y\vq{j}+z\vq{k}
\end{align}
其他文字其他文字其他文字其他文字其他文字其他文字
\begin{align}
    \vq{r}=x\vq{i}+y\vq{j}+z\vq{k}\\
    \vq{r}=x\vq{i}+y\vq{j}+z\vq{k}
\end{align}
