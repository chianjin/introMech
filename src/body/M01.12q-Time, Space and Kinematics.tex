\questions
\fangsong
\question 一个原则上不能进行直接或间接测量的物理量是否有意
义?

\question 平均速率有两种意思,一是指平均速度矢量的大小,一
是指物体运动路径总长度除以所用的总时间。这两种意思是否相
同?

\question 质点作一维运动,如果加速度不是恒量,质点的平均速率
是否等于$\dfrac 1 2$(初速+末速)?

\question 当物体的加速度恒定不变时,它的运动方向可否改变?

\question 质点的运动方程为$x=x(t)$,$y=y(t)$。在计算它的速度和
加速度的大小时,有人先求出$r=\sqrt{x^2+y^2}$,然后根据$v=\dfrac{dr}{dt}$,
$a=\dfrac{d^2r}{dt^2}$,求得结果;有人先计算速度和加速度分量,再合成,所
得结果为:
\begin{equation*}
    \begin{aligned}
        v &=\sqrt{\left(\frac{d x}{d t}\right)^{2}+\left(\frac{d y}{d t}\right)^{2}} \\
        a &=\sqrt{\left(\frac{d^{2} x}{d t^{2}}\right)^{2}+\left(\frac{d^{2} y}{d t^{2}}\right)^{2}}
    \end{aligned}
\end{equation*}
你认为哪一组结果正确?

\question  两千多年前,住在尼罗河口亚历山大城的埃拉托色尼,首
估算出地球的半径。然而,真正沿着地球子午线用绳长及日晷对
地球进行测量的,却是中国唐代高僧一行(公元683$\sim$727)。你知
道高僧一行测量的原理,方法及计算结果吗?

\question  设想将一小球上抛。如不考虑空气阻力,试证明它返回原
地时的速率等于开始的速率,并证明上升和下落所经过的时间相
等。

\question  将一小球铅直地上抛,若考虑空气阻力,它上升和下降所
经过的时间哪一个长?

\question  假设有两石块$m$和$M$,其中$m$较轻,$M$较重。按照亚里士多
德的看法,在地面上$M$应该比$m$下落得快些。伽利略首先用思辨
的方式指出亚里士多德的看法是自相矛盾的。伽利略说,设想将
$m$与$M$系在一起,则构成物体$(m+M)$,此物下落时,因为$m$有下
落得较慢的趋势,所以,$m$应该阻碍$M$,使它下落得比$m$快些而又
比$M$慢些;但是另一方面,物体$(m+M)$比$M$还要重,按照亚里
士多德的看法,它应比$M$下落得更快。因此,导致矛盾。
你认为伽利略的推理是否正确?

\question  质点在空间的运动,一般是三维运动。忽略风的作用的抛
体运动,为什么可以作为二维运动来处理?
\normalfont
