\section{伽利略变换}\label{sec:02.05}

我们可以把前两节的结果概括如下,对于任何一组相互作匀

\noindent 速运动的参考系而言,速度是相对的,即同一质点相对于不同参
考系有不同的速度;加速度是绝对的,即同一质点相对于不同参
考系的加速度是一样的。

这些结果是相当平凡的,由日常生活的经验也不难接受这些
结果,它们似乎很“浅显”。然而,物理学的特点之一,就是不
放过任何一个“浅显”的概念,总是力图找出这些“浅显”概念
的根基是什么。上两节的目的,正是要找出由式\eqref{eqn:02.02.06}所表达
的速度相对性,和由式\eqref{eqn:02.04.04}所表达的加速度绝对性的根基。

它们的根基就是参考系$K$与$K'$的时空度量之间的变换关系。
现在,我们较仔细地分析一下这个变换关系。

仍假定$K$与$K'$二者相对作匀速运动,$K$的时空坐标为$t$,$\vec{r}$;$K'$
的为$t'$,$\vec{r}'$。$K$与$K'$的时空坐标之间的关系之一,由式\eqref{eqn:02.03.03}给
出,即
\begin{equation}\label{eqn:02.05.01}
    \vec{r}'=\vec{r}-\vec{u} t -\vec{d}_0
\end{equation}
其中$\vec{u}$是$K'$相对于$K$的运动速度。
在推导式\eqref{eqn:02.03.06}或式\eqref{eqn:02.04.04}时,我们还隐含地应用过另外
一个关系。注意在式\eqref{eqn:02.03.05}中我们用了
\begin{equation}\label{eqn:02.05.02}
    \vec{v}'=\frac{\dif \vec{r}'}{\dif t}
\end{equation}
它作为质点相对于$K'$的速度,然而,严格地说,$\vec{v}'$应定义为
\begin{equation}\label{eqn:02.05.03}
    \vec{v}'=\frac{\dif \vec{r}'}{\dif t'}
\end{equation}
即必须用$K'$系的时间$t'$。这样,对比式\eqref{eqn:02.05.02}及式\eqref{eqn:02.05.03}就可
以看到,在这里我们实质上假定$\dif t'=\dif t$,或者
\begin{equation}\label{eqn:02.05.04}
    t'=t+t_0
\end{equation}
其中,$t_0$为一常数。

式\eqref{eqn:02.05.01}及式\eqref{eqn:02.05.04}给出了$K$与$K'$的时空坐标之间的完整
的变换关系,它被称为伽利略变换。式\eqref{eqn:02.03.06}及式\eqref{eqn:02.04.04}都只
\clearpage
\noindent 是伽利略变换的推论。

伽利略变换表明,时间、空间具有下列的基本性质。

\heiti 1. 时间间隔的绝对性 \normalfont

对于一个运动过程,相对于$K$,它的开始与终了的时刻若分
别为$t_1$,$t_2$,则相对于$K'$它的开始与终了的时刻分别为$t_1'=t_1+t_0$,
$t_2'=t_2+t_0$,因此有
\begin{equation}\label{eqn:02.05.05}
    \Delta t \equiv t_2 - t_1 = t_2' - t_1' \equiv \Delta t'
\end{equation}
上式的物理意义是,一个过程的时间间隔与参考系的选取无关,
是绝对的。

\heiti 2. 长度的绝对性 \normalfont

有任一直尺,相对于$K$,它的两个端点的坐标为$\vec{r}_1$,$\vec{r}_2$,则相
对于$K'$,端点的坐标应分别是$\vec{r}_1' = \vec{r}_1 - \vec{u} t - \vec{d}_0$,$\vec{r}_2' = \vec{r}_2 - \vec{u} t - \vec{d}_0$,
故有
\begin{equation}\label{eqn:02.05.06}
    |\vec{r}_1 - \vec{r}_2| = |\vec{r}'_1 - \vec{r}'_2|
\end{equation}
它的物理意义是,一直尺的长度是与参考系的选取无关的,是绝
对的。

总之,速度相对性是以时间间隔和长度的绝对性为基础的,
